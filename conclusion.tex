% conclusion.tex
\documentclass[dareport.tex]{subfiles}
\begin{document}
% Content here
\section{Conclusion}

Contrast to single machine algorithm, the designing of distributed algorithm involves an extensive consideration on communication protocol as algorithm steps --- which to say, to initialize variable \emph{X} in the distributed algorithm, the process might start with its peer process event notification to set the \emph{X} value. This also highlights that the choice of group communication pattern which lay the foundation of developing distributed algorithms such as Multicast, Broadcast or Unicast.

Furthermore, it is inevitable to make use of \textbf{\emph{timeout}} factor for orchestrating of  distributed processes coordination. This is observed in such that even a reliable connection never guarantee successful delivery of message to the destined application process. In such case, the \textbf{\textit{timeout}} factor on the coordinating process to make sure the terminal condition of the distributed algorithm has reached.

In our final experimentation deployment on Strike System with 4 server-nodes cluster setup, we observe the eventual consistency of how these algorithms work together to reach reliable fault-tolerant consistent state. This is in deed a surprise finding for us. The fact that these algorithms works on timed-interval and timeout parameters, a careful fine tuning of these parameters give how we like to run the cluster in a specify fault-tolerant criterion and boundary. It also enlighten us on how we can apply a combination of these algorithms on anther type of distributed applications development such as database cluster, data replication server, distributed game system and so on.

By attempting on developing the chosen set of distributed algorithms --- Failure Detection, Leader Election, Consensus as well as the choice of Strike Chat Messaging Application --- we achieve the learning objective of this project and able to demonstrate that we have implemented the chosen algorithms in most realistic setting.


\end{document}