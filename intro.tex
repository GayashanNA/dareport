% intro.tex
\documentclass[dareport.tex]{subfiles}
\begin{document}
% Content here

\section{Introduction}
In this report, we present the development of Distributed Multi-Server Chat System --- codename \textbf{Strike} --- which we discuss especially in the context of Distributed Algorithms. The objective of the project is to acquire learning experiences by critically analysing, experimenting and finally implementing a set of peer-reviewed publications on distributed algorithms in real world setting.
%a realistic distributed application.

%\subsection{Background}
%This report is prepared for the term project assessment of \textit{COMP90020 Distributed Algorithms Semester 1 2017, The University of Melbourne}. 

%\subsection{Idea and Approach}
To achieve the project objectives, we start by brainstorming a possible application that can give us a challenging opportunity to touch base with Distributed Algorithms. We have gone through a couple of proposed ideas and, decided to develop a Chat System which comprises of M-number of clients and N-number of servers where $ M,N \in \{1, 2, 3,\dots\} $. We hypothesize that building many clients to many servers and, server to server communication within a cluster can produce great learning exposure to demonstrate developing Distributed Systems and Distributed Algorithms. 
% as in most possible real world setting.

%In the following sections, we discuss how we achieve the project objective by discussing the survey, %comparative analysis and implementation details about Application Logic and System Architecture, %Protocol Design, Communication Sub-system, Multi-threading and Concurrency.

In the following sections, we start our discussion on the \textbf{Application Logic} where we lay the foundation for the Strike System. Then we move on to discuss our main topics on \textbf{Failure Detection}, \textbf{Leader Election} and \textbf{Consensus} algorithms. At the end, we conclude with the observation of hybrid approach of these three algorithms to demonstrate how one can implement reliable fault-tolerance and eventual consistency in distributed systems. The \textbf{Appendix} contains the figures and pseudocodes for the algorithms.

\end{document}