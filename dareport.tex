%% Use MiKTeX to compile
%% Distributed Algorithms Term Project Report
%%
%% Overall Document Format
\documentclass[11pt]{article}
\usepackage{dareport}
\sloppy
%\setlength\footskip{28pt}
%\setlength\textheight{9.0in}
\setlength\titlebox{1.0in}

%% START PUT ANY \usepackage{coo_package} AND ITS SETTINGS HERE

\usepackage{enumitem}
\usepackage{subfiles}

%% Maths
\usepackage{mathtools}

%% Plot
\usepackage{tikz}
\usepackage{pgfplots}
\pgfplotsset{width=8cm,compat=1.9}
\usetikzlibrary{arrows}

%% Algorithm
\usepackage{amsmath}
\usepackage{algorithm}
\usepackage[noend]{algpseudocode}
\makeatletter
\def\BState{\State\hskip-\ALG@thistlm}
\makeatother

%% Images
\usepackage{graphicx}
\graphicspath{ {images/} }

%% Reference
\usepackage[backend=bibtex,style=numeric,sorting=none]{biblatex}
\addbibresource{dareport.bib}

%% For intelligent referencing sections, labels, etc.
\usepackage{cleveref}

%% Title/Authors
\title{Distributed Multi-Server Chat System}

\author{San Kho Lin \\ \texttt{(829463)} \And
Gayashan N. Amarasinghe \\ \texttt{(827864)} \And
Yixin Chen \\ \texttt{(522819)} \AND
(sanl1, gamarasinghe, alachen)@student.unimelb.edu.au}

\begin{document}
\maketitle


%% START MAIN CONTENT HERE

%\begin{abstract}
%Contrast to single machine algorithm, the design of distributed algorithm involves an extensive %consideration on communication channel as algorithm steps --- which to say, to initialize variable x %in the distributed algorithm, the process might start with its peer process event communication to %set the x value!
%\end{abstract}

%\begin{keywords} 
%distributed algorithms, gossip, fast-bully, leader election, consensus, eventual consistency, %reliable communication, messaging protocol, multicast, udp, tcp, JMPP
%\end{keywords}


%% App Logic Section
\subfile{intro}

%% App Logic Section
\subfile{applogic}

%% Failure Detection Section
\subfile{failuredetection}

%% Election Section
\subfile{election}

%% Consensus Section
\subfile{consensus}

%% Page Break
%\newpage

%% Section
%\subfile{sample}

%% Page Break
\newpage

%% Conclusion Section
\subfile{conclusion}

\newpage

%% References
\section{References}

%\printbibheading
%\printbibliography[heading=none]
\printbibliography[nottype=online,heading=subbibliography,title={Paper and Printed Sources}]
\printbibliography[type=online,heading=subbibliography,title={Online Sources}]

\newpage

%% Appendix
\subfile{appendix}


\end{document}
